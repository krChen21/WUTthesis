%---------------------------------------------------------------------------%
%->> Frontmatter
%---------------------------------------------------------------------------%
%-
%-> 生成封面
%-
\maketitle% 生成中文封面
\MAKETITLE% 生成英文封面
%-
%-> 作者声明
%-
\makedeclaration% 生成声明页
%-
%-> 中文摘要
%-
\intobmk\chapter*{摘\quad 要}% 显示在书签但不显示在目录
\setcounter{page}{1}% 开始页码
\pagenumbering{Roman}% 页码符号

本研究综合评述人工智能(AI)技术的关键演进路径与应用前沿,突显了机器学习作为核心驱动力的角色,以及深度学习带来的革命性进展。

在第一章,本文深入剖析深度学习模型,包括CNN、RNN及其衍生结构、GANs,揭示这些模型在图像与语音处理等领域的显著能力。进一步探讨机器学习算法的最新进展,涵盖强化学习、集成学习、聚类与降维技术,这些创新为AI进步奠定了坚实基础。

第二章着重研究深度学习、联邦学习、强化学习的研究热点与潜力方向,并分析了其应用前景。强调了大数据与AI的共生关系,探讨了数据挖掘及特征工程对模型性能优化的重要性,并指出了TensorFlow、PyTorch等深度学习框架及AutoML工具在促进技术民主化方面的作用。针对深度学习、联邦学习与强化学习的前沿议题,本综述剖析了深度学习的泛化能力和可解释性挑战、联邦学习框架下的隐私保护机制、以及强化学习在决策智能领域的进展与局限,同时展望了小样本与零样本学习在缓解数据瓶颈方面的潜力。

通过对比分析,本文以表格形式系统总结了上述三种学习方法的核心特性,并概述其在医疗健康、科学研究、娱乐媒体、智能汽车、金融科技、在线教育等多个行业的广泛应用情况。综上,本文系统性回顾了AI领域的最新研究成果与技术动向以及应用前景。

\keywords{深度学习,联邦学习,强化学习,可解释性,应用前景}% 中文关键词
%-
%-> 英文摘要
%-
\intobmk\chapter*{Abstract}% 显示在书签但不显示在目录

This study provides a comprehensive review of the key evolutionary paths and application frontiers of Artificial Intelligence (AI) technologies, highlighting the role of machine learning as a core driver and the revolutionary advances brought about by deep learning.

In the first chapter, this paper provides an in-depth analysis of deep learning models, including CNNs, RNNs and their derived structures, and GANs, revealing the remarkable capabilities of these models in areas such as image and speech processing. It further explores the latest advances in machine learning algorithms, covering reinforcement learning, integrated learning, clustering and dimensionality reduction techniques, innovations that have laid a solid foundation for AI advancement.

Chapter 2 focuses on the research hotspots and potential directions of Deep Learning, Federated Learning, and Reinforcement Learning, and analyzes their application prospects. It emphasizes the symbiotic relationship between big data and AI, discusses the importance of data mining and feature engineering for model performance optimization, and points out the role of deep learning frameworks such as TensorFlow, PyTorch, and AutoML tools in facilitating the democratization of technology. Addressing the cutting-edge topics of deep learning, federated learning, and reinforcement learning, this review dissects the generalization ability and interpretability challenges of deep learning, the privacy-preserving mechanisms under federated learning frameworks, and the advances and limitations of reinforcement learning in the field of decision-making intelligence, while looking ahead to the potentials of small-sample and zero-sample learning in alleviating data bottlenecks.

Through comparative analysis, this paper systematically summarizes the core features of the above three learning methods in table form, and outlines their wide applications in various industries, including healthcare, scientific research, entertainment media, smart cars, financial technology, and online education. In summary, this paper systematically reviews the latest research results and technology trends in the field of AI as well as the application prospects.

\KEYWORDS{Deep Learning,Federated Learning,Reinforcement Learning,Interpretability,Application Perspectives}% 英文关键词
%---------------------------------------------------------------------------%
