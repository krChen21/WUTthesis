\chapter{引言}\label{chap:Firse}

\section{add}

AI领域\cite{1022526865.nh}的最新进展\cite{1022776046.nh}非常广泛,涉及到多个子领域\cite{1022802292.nh}和研究方向\cite{1023026619.nh}。以下\cite{1023026619.nh}是一些根据搜索\cite{1023126184.nh}结果得出的关键进展\cite{1023074826.nh}:

\subsection{小样本概念学习\cite{1023473037.nh} }

我国科研团队\cite{1023532465.nh}在小样本概念学习\cite{1023532487.nh}方面取得了突破\cite{1023736935.nh},AI系统\cite{1023742110.nh}能够在没有大量数据训练\cite{1023842249.nh}的情况下,通过概念学习\cite{1023854471.nh}和逻辑推理\cite{1023901740.nh}完成任务。这一成果\cite{1024000941.nh}在国际顶级学术期刊《科学·进展》上发表\cite{1024350451.nh},标志着我国在AI领域的研究跻身世界前列\cite{1024351780.nh}。

\subsection{AI产业规模增长\cite{1024364355.nh}} 

据艾瑞咨询统计\cite{1024366234.nh},2023年中国人工智能产业规模\cite{AGLU202211003115}已达到2,137亿元,并预计在2028年将增长至8,110亿元\cite{DQXX202307004}。AI技术\cite{JSJG202302008}已经深入医疗、教育、金融、制造等多个行业\cite{JSWL202401014}。

\subsection{AI创生时代\cite{KXSY202202006} }

张一甲在AI趋势讨论\cite{KZLY2023092701I}中提出AI创生时代\cite{LXJZ202302005}的概念,包括AI生产时代\cite{LXXB202305014}、AI原生时代、AI创生时代、AI文明时代四个阶段\cite{LXXB202305014}。AI正从辅助工具向主导角色转变\cite{QDHY202402015},映射能力逐渐增强。

\subsection{AI技术商业化\cite{RN1} }

昆仑万维等公司在AI技术领域\cite{RN2}取得显著进步\cite{RN3},如天工3.0模型\cite{RN4}的发布,展示了AI技术在搜索\cite{RN5}、内容创作\cite{RN6}等方面的应用潜力。

\subsection{通用人工智能(AGI)\cite{RN7}}

2024年,AGI\cite{RN8}的研究取得了显著进展,一些模型\cite{RN9}开始展现出接近甚至超越人类的多领域适应能力\cite{RN10},AI开始理解和解决跨领域问题\cite{RN11}。

\subsection{AI基础设施\cite{RN12}}

AI技术的商业化和产业应用需要通过降低成本来实现\cite{RN13},特别是在算力和能耗方面。AI基础设施的创新变得尤为关键,包括硬件芯片\cite{RN14}、云计算、数据处理等。

\subsection{国际AI进展\cite{RN15} }

从美国当地时间5月13日到6月10日,全球科技巨头\cite{RN16}将密集公布AI领域最新进展,预示着AI技术的新纪元即将到来\cite{RN17}。

这些进展展示了AI领域的快速发展和未来潜力\cite{RN18},从基础研究到商业应用,AI正在不断推动技术革新和产业升级\cite{RN19}。随着技术的不断进步\cite{RN20},我们可以期待AI将在更多领域发挥关键作用\cite{RN21},为社会带来更多便利和创新解决方案\cite{RN22}。